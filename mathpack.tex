\documentclass{article}
\usepackage{ucs}
\usepackage[T2A]{fontenc}
\usepackage[utf8x]{inputenc}
\usepackage[russian, english]{babel}
\usepackage{listings}
\usepackage{graphicx}
\usepackage{amsmath}
\author{Гвоздев П.А.}
\title{SciPy}
\date{27 Мая 2023}



\usepackage{xcolor}

%New colors defined below
\definecolor{codegreen}{rgb}{0,0.6,0}
\definecolor{codegray}{rgb}{0.5,0.5,0.5}
\definecolor{codepurple}{rgb}{0.58,0,0.82}
\definecolor{backcolour}{rgb}{0.95,0.95,0.92}
\lstdefinestyle{mystyle}{
  backgroundcolor=\color{backcolour}, commentstyle=\color{codegreen},
  keywordstyle=\color{magenta},
  numberstyle=\tiny\color{codegray},
  stringstyle=\color{codepurple},
  basicstyle=\ttfamily\footnotesize,
  breakatwhitespace=false,         
  breaklines=true,                 
  captionpos=b,                    
  keepspaces=true,                 
  numbers=left,                    
  numbersep=5pt,                  
  showspaces=false,                
  showstringspaces=false,
  showtabs=false,                  
  tabsize=2
  }
  \lstset{style=mystyle}


\begin{document}
\maketitle
\newpage
SciPy - это набор математических алгоритмов и удобных функций, построенный на библиотеке NumPy.\\
SciPy организован в подпакеты, охватывающие различные области научных вычислений. Они обобщены в следующей таблице:
\\
\begin{center}
\begin{tabular}{ |c|c| } 
 \hline
 \textbf{Подпакет} & \textbf{Описание}  \\
 \hline
 \hline
 cluster & Алгоритмы кластеризации  \\ 
  \hline
 constants & Физические и математические константы  \\ 
  \hline
 fftpack &  Процедуры быстрого преобразования Фурье \\ 
  \hline
 integrate & Решатели интегрирования и обыкновенных дифференциальных уравнений  \\ 
  \hline
 interpolate & Интерполяция и сглаживание сплайнов  \\ 
  \hline
 io & Ввод и вывод  \\ 
  \hline
 linalg & Линейная алгебра \\ 
  \hline
 ndimage & Обработка N-мерных изображений \\ 
  \hline
 odr & Регрессия ортогонального расстояния  \\ 
  \hline
 optimize & Процедуры оптимизации и поиска корня  \\ 
  \hline
 signal & Обработка сигналов  \\ 
  \hline
 sparse & Разреженные матрицы и связанные с ними подпрограммы  \\ 
  \hline
 spatial & Пространственные структуры данных и алгоритмы  \\ 
  \hline
 special & Специальные функции  \\ 
  \hline
 stats & Статистические распределения и функции \\ 
 \hline
\end{tabular}
\end{center}
\\
Далее рассмотрим некоторые из них.
\begin{section}{Интегральные функции}
Подпакет предоставляет несколько методов интегрирования, включая интегратор обыкновенных дифференциальных уравнений. Возвращаемое значение большинства функций представляет собой кортеж, в котором первый элемент содержит оценочное значение интеграла, а второй элемент содержит верхнюю границу ошибки.
\newpage
Функция \textbf{quad} предназначена для интеграции функции одной переменной на заданном отрезке. Например, предположим, что вы хотите вычислить вот такой интеграл.
\begin{align*}
I(a,b)=\int_{0}^{1} ax^2+b \,dx
\end{align*}  
\\Вычислим значение этой функции в точке (2 ; 1).
\begin{lstlisting}[language=Python]
from scipy.integrate import quad
def integrand(x, a, b):
    return a*x**2 + b
a = 2
b = 1
I = quad(integrand, 0, 1,args=(a,b))
I

>>(1.6666666666666667, 1.8503717077085944e-14)
\end{lstlisting}
\\
\\
Механизмы взятия двойных и тройных интегралов были объединены в функции \textbf{dblquad} и \textbf{tplquad}. Эти функции используют функцию для интегрирования и четыре или шесть аргументов соответственно. Пределы всех внутренних интегралов должны быть определены как функции.
В качестве примера для непостоянных пределов рассмотрим интеграл\\
\begin{align*}
I=\int_{0}^{1}\,dy\int_{1-3y}^{1+2y} x^2y \,dx  
\end{align*}

 Его можно вычислить так:
 \\
 \begin{lstlisting}[language=Python]
from scipy.integrate import dblquad
I = dblquad(lambda x, y: x**2*y, 0, 1, lambda y: 1-3*y, lambda y : 1+2*y)
I

>>(2.75, 1.2852467283249305e-13)
\end{lstlisting}
\end{section}
\begin{section}{Интерполяция}


В SciPy доступно несколько общих возможностей для интерполяции и сглаживания данных в 1, 2 и более высоких измерениях. Выбор конкретной процедуры интерполяции зависит от данных: являются ли они одномерными, представлены в структурированной сетке или неструктурированными. Еще одним фактором является желаемая плавность интерполятора.
\\
Если все, что вам нужно, это линейная (или прерывистая линия) интерполяция, вы можете использовать процедуру \textbf{numpy.interp} Для интерполяции требуется два массива данных, $x$ и $y$, и третий массив точек, $xnew$ для оценки интерполяции.
\\
\begin{lstlisting}[language=Python]
x = np.linspace(0, 10, num=11)
y = np.cos(-x**2 / 9.0)

xnew = np.linspace(0, 10, num=1001)
ynew = np.interp(xnew, x, y)

plt.plot(xnew, ynew, '-', label='linear interp')
plt.plot(x, y, 'o', label='data')
plt.legend(loc='best')
plt.show()
\end{lstlisting}
\\
\includegraphics[width=0.8\linewidth]{interpol1.png}
\\
Конечно, кусочно-линейная интерполяция создает углы в точках данных, где соединяются линейные фрагменты. Для создания более плавной кривой можно использовать кубические сплайны, где интерполирующая кривая состоит из кубических фрагментов с соответствующими первой и второй производными. В коде эти объекты представлены через \textbf{CubicSpline} экземпляры класса. Экземпляр создается с использованием $x$ и $y$ массивов данных, а затем его можно оценить, используя целевые $xnew$ значения:
\\
\begin{lstlisting}[language=Python]
from scipy.interpolate import CubicSpline
x = np.linspace(0, 10, num=11)
y = np.cos(-x**2 / 9.)
spl = CubicSpline(x, y)
fig, ax = plt.subplots(4, 1, figsize=(5, 7))
xnew = np.linspace(0, 10, num=1001)
ax[0].plot(xnew, spl(xnew))
ax[0].plot(x, y, 'o', label='data')
ax[1].plot(xnew, spl(xnew, nu=1), '--', label='1st derivative')
ax[2].plot(xnew, spl(xnew, nu=2), '--', label='2nd derivative')
ax[3].plot(xnew, spl(xnew, nu=3), '--', label='3rd derivative')
for j in range(4):
    ax[j].legend(loc='best')
plt.tight_layout()
plt.show()
\end{lstlisting}
\\
\includegraphics[width=0.8\linewidth]{interpol2.png}
\end{section}
\\
\begin{section}{Оптимизация}
\textbf{scipy.optimize} предстявляет набор функций, которые реализуют популярные алгоритмы оптимизаци: 
\begin{itemize}
    \item Неограниченная и ограниченная минимизация многомерных скалярных функций (например, Алгоритм Бройдена — Флетчера — Гольдфарба — Шанно, метод сопряженных градиентов, метод Нелдера — Мида).
    \item Глобальная оптимизация (дифференциальная эволюция, двойной отжиг и т. д.).
    \item Минимизация наименьших квадратов и подбор кривой (метод наименьших квадратов, приближение с помощью кривых);
    \item Минимизаторы скалярных одномерных функций и численное решение уравнений.
    \item Функции для решения систем многомерных уравнений с помощью таких алгоритмов, как Пауэлла, Левендберга — Марквардта и ещё куча всего.
\end{itemize}
\\
Пусть у нас есть функция $f(x)=x^2+10\sin(x)$. Найдём её минимум. Можно использовать алгоритм поиска методом перебора, который будет оценивать каждую точку в сетке диапазонов. Если область определения функции достаточно большая, то \textbf{brute()} становится очень медленным. Можно взять функцию \textbf{scipy.optimize.anneal()}
\\
\begin{lstlisting}[language=Python]
from scipy import optimize

def target_function(x):
    return x ** 2+10*np.sin(x)
 
plt.figure(figsize=(5,5))
x = np.linspace(-10, 10, 1000)
plt.xlabel('x')
plt.ylabel('y')
plt.title('optimize')
plt.plot(x, target_function(x),  label=r'$f(x)=x^2+10\sin(x)$')


grid = (-10, 10)
xmin_global = optimize.brute(target_function, (grid,))
print(xmin_global)

a = target_function(xmin_global)
plt.plot(xmin_global, a, 'o', label='min')
plt.legend()
plt.grid()
plt.show()

>>[-1.30640933]
\end{lstlisting}
\includegraphics[width=0.8\linewidth]{optimum1.png}
\\
На последок посомтрим на поиск корней. Для нахождения кореней уравнений используется функция scipy.optimize.root(fun, x0, args=(), method, ..., где:
\begin{itemize}
    \item fun - целевая функция.
    \item x0 - начальное.
    \item args - дополнительные аргументы, которые передаются целевой функции и ее якобиану.
    \item method - метод решения.
\end{itemize}
\\
Решим уравнение $x^2-2\cos(x)=0$ с начальным значением $x_0=0.3$.
\\
\begin{lstlisting}[language=Python]
from scipy.optimize import root

def target_function(x):
   return x ** 2 - 2 * np.cos(x)

x0 = 0.3
x_root=root(target_function, x0).x
print(x_root)

plt.figure(figsize=(5,5))
x = np.linspace(-2, 2, 100)
plt.xlabel('x')
plt.ylabel('y')
plt.plot(x, target_function(x),  label=r'$f(x)=x^2-2\cos(x)$')
plt.plot(x_root, target_function(x_root), 'o', label='root')
plt.legend()
plt.grid()
plt.show()

>>[1.02168995]
\end{lstlisting}
\includegraphics[width=0.8\linewidth]{optimum2.png}

\end{section}
\end{document}

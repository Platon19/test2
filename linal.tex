\documentclass{beamer}
\usepackage[russian]{babel}
\usepackage{ucs}
\usepackage[T2A]{fontenc}
\usepackage{mathtools}
\setbeamertemplate{footline}[frame number]
\usetheme{Goettingen}
\usecolortheme{whale}
\usepackage{graphicx}%Вставка картинок правильная

\usepackage{float}%"Плавающие" картинки

\usepackage{wrapfig}%Обтекание фигур (таблиц, картинок и прочего)

%\usepackage[utf8x]{inputenc}


\title[Линейное аффинное многообразие]{\textbf{Линейное аффинное многообразие}}
%\subtitle{Subtitle}

\author{Гвоздев П.А ФН1-21Б}

\date{24 мая 2023 г.} 

\begin{document}

\frame{\titlepage}


\begin{frame}{\textbf{Основные понятия}}

\begin{block}{Определение}
Пусть $\mathbf{V}$ произвольное линейное пространство\\
над полем $\mathfrak{K}$ и $\mathbf{H}=x_0+\mathbf{L}$-смежный класс по подпространству $\mathbf{L}\subset \mathbf{V}$ ($x_0$-фиксированный  вектор из $\mathbf{V}$). Говорят, что $\mathbf{H}$-линейное аффинное многообразие пространства $\mathbf{V}$.
\end{block}

\begin{block}{Замечание}
Подпространство $\mathbf{L}$ называют направляющим подпространством,
а вектор $x_0$-вектором сдвига.
\end{block}


\end{frame}


\begin{frame}{\textbf{Основные понятия}}

\begin{block}{Определение}
Размерностью $\mathbf{H}$ называется размерность соответствующего направляющего подпространства $\mathbf{L}$.
\end{block}
\begin{block}{Замечание}
Вектор сдвига определён неоднозначно.
\end{block}
\begin{block}{Примеры}

\begin{itemize}
\item Если $\dim(H)=0$ - то $\mathbf{H}$ называется, естественно, точкой 
\item При $\dim(H)=1$ - прямой
\item При $\dim(H)=\dim(\mathbf{V})-1$, $\mathbf{H}$ называется гиперплоскостью
\end{itemize}
 
\end{block}

\end{frame}



\begin{frame}{\textbf{Некоторые следствия из алгебры}}


\begin{block}{Следствие 1}
Пусть \mathbf{V}-линейное пространство, $\mathbf{L}\subset \mathbf{V}$- его подпространство. Тогда \mathbf{V} распадается на непересекающиеся линейный аффинные многообразия с направляющим подпространством $\mathbf{L}$.
\end{block}


\begin{block}{Следствие 2}
 Так как ($\mathbf{V}$,+) является абелевой группой, то любая подгруппа нормальная, и можно определеить структуру фактор-группы
 $\mathbf{V}/\mathsf{L}=$\{$\mathsf{L},\mathbf{x_1}+\mathsf{L},\mathbf{x_2}+\mathsf{L},..,\mathbf{x_n}+\mathsf{L},..$\} с бинарной операцией $\mathbf{a}+\mathsf{L}\circ \mathbf{b}+\mathsf{L}=(\mathbf{a}+\mathbf{b})+\mathsf{L}$. Очевидно, введённая операция не будет зависеть от выбора представителей в классах. Таким образом можно ввести операцию сложения линейных аффинных многообразий с одним и тем же направляющим подпространством.
\end{block}


\end{frame}


\begin{frame}{\textbf{Некоторые следствия из алгебры}}
\begin{figure}[h]

\centering

\includegraphics[width=0.8\linewidth]{image.png}

%\caption{Иллюстрация к следствию 2}

Иллюстрация к следствию 2
\label{fig:mpr}

\end{figure}

\end{frame}


\begin{frame}{\textbf{Ещё немного теории}}
\begin{block}{Определение}
    Два многообразия $\mathbf{H_1}=a+\mathbf{L_1}$ и $\mathbf{H_2}=b+\mathbf{L_2}$ называютя параллельными, если $\mathbf{L_1} \subset \mathbf{L_2}$ или $\mathbf{L_2} \subset \mathbf{L_2}$.
\end{block}
\begin{block}{Теорема}
    Если 2 параллельных многообразия пересекаются, то 1 из них содержит другое.
\end{block}
\begin{block}{Доказательство:}
Пусть $x_0\in\mathbf{H_1}\cap\mathbf{H_2}$. Тогда $\mathbf{H_1}=x_0+\mathbf{L_1}$ и $\mathbf{H_2}=x_0+\mathbf{L_2}$. Пусть $\mathbf{L_1} \subset \mathbf{L_2}$, тогда $\mathbf{H_2}=x_0+\mathbf{L_1}+(\mathbf{L_2}\setminus \mathbf{L_1})=\mathbf{H_1}+\mathbf{L_1}^\delta
\Rightarrow \mathbf{H_1} \subset \mathbf{H_2}$.
\end{block}
\end{frame}


\begin{frame}{\bfПересечение ЛАМ с подпространством, дополнительным к направляющему подпространству}

\begin{block}{Определение}
    $\mathbf{L}^\delta$ называется прямым дополнением $\mathbf{L}$ до $\mathbf{V}$, если $\mathbf{L}\oplus \mathbf{L}^\delta=\mathbf{V}$.
\end{block}

\begin{block}{Змаечание}
    Однозначно определена только $\dim\mathbf{L}^\delta$.
\end{block}

\begin{block}{Теорема}
 Пересечение линейного многообразия с подпространством, дополнительным к его направляющему подпространству состоит ровно из 1 вектора.   
\end{block}

\end{frame}



\begin{frame}{\bfПересечение ЛАМ с подпространством, дополнительным к направляющему подпространству}
\begin{block}{Доказательство}
    Пусть $\mathbf{V}=\mathbf{L}\oplus \mathbf{L}^\delta$, $\mathbf{H}=\mathbf{x_0}+\mathbf{L}$. Тогда $\forall\:\mathbf{x} \in \mathbf{V},    \mathbf{x}=\mathbf{y}+\mathbf{z}$, где $\mathbf{y}\in\mathbf{L}$, $\mathbf{z}\in\mathbf{L}^\delta$.
    Следовательно, $\exists\:\mathbf{y_0}\in\mathbf{L}$ и $\exists\:\mathbf{z_0}\in\mathbf{L}^\delta:\mathbf{x_0}=\mathbf{y_0}+\mathbf{z_0}\Rightarrow\mathbf{z_0}=\mathbf{x_0}+(-\mathbf{y_0})$. Очевидно, $\mathbf{z_0}\in\mathbf{H}\cap\mathbf{L}^\delta$, тогда остаётся доказать единственность. Допустим противоположное, $\exists\: \mathbf{t_0}\neq\mathbf{z_0}:\mathbf{t_0}\in\mathbf{H}\cap\mathbf{L}^\delta$ . Выберем другого представителя класса сопряжённости, то есть $\mathbf{z_0}+\mathbf{L}=\mathbf{x_0}+(-\mathbf{y_0})+\mathbf{L}=\mathbf{x_0}+\mathbf{L}=\mathbf{H}$.
    Тогда
    $\exists\:\mathbf{\ell }\in\mathbf{L}:\mathbf{t_0}=\mathbf{z_0}+\mathbf{\ell }\Rightarrow \mathbf{\ell}=\mathbf{t_0 }-\mathbf{z_0}\in\mathbf{L}^\delta$.Очевидно, что $\mathbf{\ell }=\theta\Rightarrow\mathbf{t_0}=\mathbf{z_0}$-противоречие.
\end{block}

\end{frame}



\begin{frame}{Список литературы}
  
    [1]А.Н. Канатников, А.П. Крищенко. {\bf Линейная алгебра}, ред. В.С. Зарубин, А.П. Крищенко
    - 3-е издание, изд. МГТУ им. Н.Э. Баумана, 2002.
    
    [2]А.И. Кострикин {\bf Введение в алгебру, Ч. II: Линейная алгебра}- Четвёртое издание, стереотип. - М.:МЦНМО, 2021.
\end{frame}

\end{document}
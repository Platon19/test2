\documentclass{article}
\usepackage{ucs}
\usepackage[T2A]{fontenc}
\usepackage[utf8x]{inputenc}
\usepackage[russian, english]{babel}
\author{Гвоздев П.А.}
\title{Отчёт по графикам}
\date{23 Мая 2023}

\begin{document}
\maketitle
\documentclass{article}
\usepackage{ucs}
\usepackage[T2A]{fontenc}
\usepackage[utf8x]{inputenc}
\usepackage[russian, english]{babel}
\usepackage{listings}
\usepackage{graphicx}
\usepackage{amsmath}
\author{Гвоздев П.А.}
\title{SciPy}
\date{27 Мая 2023}



\usepackage{xcolor}

%New colors defined below
\definecolor{codegreen}{rgb}{0,0.6,0}
\definecolor{codegray}{rgb}{0.5,0.5,0.5}
\definecolor{codepurple}{rgb}{0.58,0,0.82}
\definecolor{backcolour}{rgb}{0.95,0.95,0.92}
\lstdefinestyle{mystyle}{
  backgroundcolor=\color{backcolour}, commentstyle=\color{codegreen},
  keywordstyle=\color{magenta},
  numberstyle=\tiny\color{codegray},
  stringstyle=\color{codepurple},
  basicstyle=\ttfamily\footnotesize,
  breakatwhitespace=false,         
  breaklines=true,                 
  captionpos=b,                    
  keepspaces=true,                 
  numbers=left,                    
  numbersep=5pt,                  
  showspaces=false,                
  showstringspaces=false,
  showtabs=false,                  
  tabsize=2
  }
  \lstset{style=mystyle}


\begin{document}
\maketitle
\newpage
SciPy - это набор математических алгоритмов и удобных функций, построенный на расширении Python NumPy.\\
SciPy организован в подпакеты, охватывающие различные области научных вычислений. Они обобщены в следующей таблице:
\\
\begin{center}
\begin{tabular}{ |c|c| } 
 \hline
 \textbf{Подпакет} & \textbf{Описание}  \\
 \hline
 \hline
 cluster & Алгоритмы кластеризации  \\ 
  \hline
 constants & Физические и математические константы  \\ 
  \hline
 fftpack &  Процедуры быстрого преобразования Фурье \\ 
  \hline
 integrate & Решатели интегрирования и обыкновенных дифференциальных уравнений  \\ 
  \hline
 interpolate & Интерполяция и сглаживание сплайнов  \\ 
  \hline
 io & Ввод и вывод  \\ 
  \hline
 linalg & Линейная алгебра \\ 
  \hline
 ndimage & Обработка N-мерных изображений \\ 
  \hline
 odr & Регрессия ортогонального расстояния  \\ 
  \hline
 optimize & Процедуры оптимизации и поиска корня  \\ 
  \hline
 signal & Обработка сигналов  \\ 
  \hline
 sparse & Разреженные матрицы и связанные с ними подпрограммы  \\ 
  \hline
 spatial & Пространственные структуры данных и алгоритмы  \\ 
  \hline
 special & Специальные функции  \\ 
  \hline
 stats & Статистические распределения и функции \\ 
 \hline
\end{tabular}
\end{center}
\\
Далее рассмотрим некоторые из них.
\begin{section}{Integrate}
Подпакет предоставляет несколько методов интегрирования, включая интегратор обыкновенных дифференциальных уравнений. Возвращаемое значение большинства функций представляет собой кортеж, в котором первый элемент содержит оценочное значение интеграла, а второй элемент содержит верхнюю границу ошибки.\\
Функция quad предназначена для интеграции функции одной переменной на заданном отрезке. Например, предположим, что вы хотите вычислить вот такой интеграл.
\begin{align*}
I(a,b)=\int_{0}^{1} ax^2+b \,dx
\end{align*}  
\\Вычислим значение этой функции в точке (2 ; 1).
\begin{lstlisting}[language=Python]
from scipy.integrate import quad
def integrand(x, a, b):
    return a*x**2 + b
a = 2
b = 1
I = quad(integrand, 0, 1,args=(a,b))
I
>>(1.6666666666666667, 1.8503717077085944e-14)
\end{lstlisting}
Механизмы взятия двойных и тройных интегралов были объединены в функции dblquad и tplquad. Эти функции используют функцию для интегрирования и четыре или шесть аргументов соответственно. Пределы всех внутренних интегралов должны быть определены как функции.
В качестве примера для непостоянных пределов рассмотрим интеграл
\begin{align*}
I=\int_{0}^{1}\,dy\int_{1-3y}^{1+2y} x^2y \,dx  
\end{align*}
\end{section}
 Его можно вычислить так:
 \\
 \begin{lstlisting}[language=Python]
from scipy.integrate import dblquad
I = dblquad(lambda x, y: x**2*y, 0, 1, lambda y: 1-3*y, lambda y : 1+2*y)
I
>>(2.75, 1.2852467283249305e-13)
\end{lstlisting}
\end{document}


\end{document}

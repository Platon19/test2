\documentclass{article}
\usepackage{ucs}
\usepackage[T2A]{fontenc}
\usepackage[utf8x]{inputenc}
\usepackage[russian, english]{babel}
\author{Гвоздев П.А.}
\title{Отчёт по графикам}
\date{23 Мая 2023}

\begin{document}
\maketitle
\documentclass{article}
\usepackage{ucs}
\usepackage[T2A]{fontenc}
\usepackage[utf8x]{inputenc}
\usepackage[russian, english]{babel}
\usepackage{listings}
\usepackage{graphicx}
\usepackage{amsmath}
\author{Гвоздев П.А.}
\title{SciPy}
\date{27 Мая 2023}



\documentclass{article}
\usepackage{ucs}
\usepackage[T2A]{fontenc}
\usepackage[utf8x]{inputenc}
\usepackage[russian, english]{babel}
\usepackage{listings}
\usepackage{graphicx}
\usepackage{amsmath}
\author{Гвоздев П.А.}
\title{Отчёт "Визуализация данных"}
\date{ФН1-21Б}



\usepackage{xcolor}

%New colors defined below
\definecolor{codegreen}{rgb}{0,0.6,0}
\definecolor{codegray}{rgb}{0.5,0.5,0.5}
\definecolor{codepurple}{rgb}{0.58,0,0.82}
\definecolor{backcolour}{rgb}{0.95,0.95,0.92}
\lstdefinestyle{mystyle}{
  backgroundcolor=\color{backcolour}, commentstyle=\color{codegreen},
  keywordstyle=\color{magenta},
  numberstyle=\tiny\color{codegray},
  stringstyle=\color{codepurple},
  basicstyle=\ttfamily\footnotesize,
  breakatwhitespace=false,         
  breaklines=true,                 
  captionpos=b,                    
  keepspaces=true,                 
  numbers=left,                    
  numbersep=5pt,                  
  showspaces=false,                
  showstringspaces=false,
  showtabs=false,                  
  tabsize=2
  }
  \lstset{style=mystyle}



\begin{document}
\maketitle
\newpage
 Для обработки я выбрал датасэт на тему заработных плат "Salary Data", включающий в себя
    такие данные как:\\
    Возраст, пол, уровень образования, профессия, опыт работы, заработная плата.
\\
Импортируем необходимые библиотеки, удалим лишнее, исправим опечатку, устанавливаем тему для графиков.
\begin{lstlisting}[language=Python]
import numpy as np
import pandas as pd
import matplotlib.pyplot as plt
from mpl_toolkits.mplot3d import Axes3D
import seaborn as sns

df=pd.read_csv('Salary_Data.csv')
df.dropna(inplace=True)
df.drop(df.index[df['Gender']=='Other'],inplace=True)
df['Education Level']=df['Education Level'].apply(lambda x: x.replace('phD','PhD'))
sns.set_theme()
\end{lstlisting}
\\
Далее построим график распределения заработной платы в зависимости от опыта работы.
\\
\begin{lstlisting}[language=Python]
plt.figure(figsize=(5,5))
sns.scatterplot(df,x='Years of Experience',y='Salary')
plt.show()
\end{lstlisting}
\\
\includegraphics[width=0.8\linewidth]{graph1.png}
\\


\\
График зависимости заработной платы от возраста.
\\
\begin{lstlisting}[language=Python]
plt.figure(figsize=(5,5))
sns.lineplot(df,x='Age',y='Salary')
plt.show()
\end{lstlisting}
\\
\includegraphics[width=0.8\linewidth]{graph2.png}
\newpage
\\
Построим график распределения зарплат по уровням образования.
\\
\begin{lstlisting}[language=Python]
plt.figure(figsize=(5,5))
sns.stripplot(df,y='Education Level',x='Salary')
plt.show()
\end{lstlisting}
\\
\includegraphics[width=0.8\linewidth]{graph3.png}

\\
Построим столбчатую диаграмму суммы зарплат всех специалистов с одним и тем же уровнем образования. Так как наименования столбцов не влазят сократим их названия.
\\
\begin{lstlisting}[language=Python]
df['Education Level']=df['Education Level'].apply(lambda x: x.replace('High School','School'))
df['Education Level']=df['Education Level'].apply(lambda x: x.replace("Master's Degree",'Master D'))
df['Education Level']=df['Education Level'].apply(lambda x: x.replace("Master's",'Master'))
df['Education Level']=df['Education Level'].apply(lambda x: x.replace("Bachelor's Degree",'Bach D'))
df['Education Level']=df['Education Level'].apply(lambda x: x.replace("Bachelor's",'Bachelor'))


plt.figure(figsize=(5,5))
sns.barplot(df,x='Education Level',y='Salary')
plt.show()
\end{lstlisting}
\\
\includegraphics[width=0.8\linewidth]{graph4.png}
\newpage
\\
Построим столбчатую диаграмму количества специалистов различных уровней образования.
\\
\begin{lstlisting}[language=Python]
plt.figure(figsize=(5,5))
sns.countplot(df,x='Education Level')
plt.show()
\end{lstlisting}
\\
\includegraphics[width=0.8\linewidth]{graph5.png}

\\
\newpage
Построим pairplot по гендерам.
\\
\begin{lstlisting}[language=Python]
plt.figure(figsize=(5,5))
sns.pairplot(df,hue='Gender')
\end{lstlisting}
\\
\includegraphics[width=0.8\linewidth]{graph6.png}

\\
\newpage
Построим график эмпирической кумулятивной функции распределения опыта работы.
\\
\begin{lstlisting}[language=Python]
plt.figure(figsize=(5,5))
sns.ecdfplot(data=df, x="Years of Experience")
plt.show()
\end{lstlisting}
\\
\includegraphics[width=0.8\linewidth]{graph7.png}

\newpage
\\
Построим усиковую диаграмму распределения зарплаты по уровням образования.
\\
\begin{lstlisting}[language=Python]
plt.figure(figsize=(5,5))
sns.boxplot(x='Education Level', y='Salary', data=df)
plt.show()
\end{lstlisting}
\\
\includegraphics[width=0.8\linewidth]{graph8.png}
\newpage
\\
Построим графики распределения зарплат в зависимости от пола и уровня образования.
\\
\begin{lstlisting}[language=Python]
plt.figure(figsize=(5,5))
sns.violinplot(x=df['Education Level'], y=df['Salary'],
               hue=df['Gender'], split=True)
plt.show()
\end{lstlisting}
\\
\includegraphics[width=0.8\linewidth]{graph9.png}

\\
\newpage
Построим histplot для самой популярной профессии- "Data Analyst".
\\
\begin{lstlisting}[language=Python]
plt.figure(figsize=(5,5))
df1=df[df['Job Title']=='Data Analyst']
sns.histplot(df1['Age'], kde=True, stat='count')
plt.show()
\end{lstlisting}
\\
\includegraphics[width=0.8\linewidth]{graph10.png}


\end{document}
